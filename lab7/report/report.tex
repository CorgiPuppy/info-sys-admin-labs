\documentclass[12pt, a4paper]{report}
\usepackage[top=1cm, left=0.8cm, right=0.8cm]{geometry}

\usepackage[utf8]{inputenc}
\usepackage[russian]{babel}

\usepackage{array}
\newcolumntype{M}[1]{>{\centering\arraybackslash}m{#1}}

\usepackage{hyperref}
\hypersetup{
	colorlinks,
	citecolor=black,
	filecolor=black,
	linkcolor=black,
	urlcolor=black
}

\usepackage{sectsty}
\allsectionsfont{\centering}

\usepackage{indentfirst}
\setlength\parindent{24pt}
 
\usepackage{listings}
\usepackage{xcolor}
\definecolor{codegreen}{rgb}{0,0.6,0}
\definecolor{codegray}{rgb}{0.5,0.5,0.5}
\definecolor{codepurple}{rgb}{0.58,0,0.82}
\definecolor{backcolour}{rgb}{0.95,0.95,0.92}
\lstdefinestyle{mystyle}{
    backgroundcolor=\color{backcolour},
    commentstyle=\color{codegreen},
    keywordstyle=\color{magenta},
    numberstyle=\normalsize\color{codegray},
    stringstyle=\color{codepurple},
    basicstyle=\ttfamily\footnotesize,
    breakatwhitespace=false,
    breaklines=true,
    captionpos=b,
    keepspaces=true,
    numbers=left,
    numbersep=5pt,
    showspaces=false,
    showstringspaces=false,
    showtabs=false,
    tabsize=2
}

\usepackage{graphicx}
\graphicspath{{assets/}}

\begin{document}
	\begin{titlepage}
		\begin{center}
			\large \textbf{Министерство науки и высшего образования Российской Федерации} \\
			\large \textbf{Федеральное государственное бюджетное образовательное учреждение высшего образования} \\
			\large \textbf{«Российский химико-технологический университет имени Д.И. Менделеева»} \\

			\vspace*{4cm}
			\LARGE \textbf{ОТЧЕТ ПО ЛАБОРАТОРНОЙ РАБОТЕ №7}

			\vspace*{4cm}
			\begin{flushright}
				\Large
				\begin{tabular}{>{\raggedleft\arraybackslash}p{8.85cm} p{10.8cm}}
					Выполнил студент группы КС-36: & Золотухин Андрей Александрович \\
					Ссылка на репозиторий: & https://github.com/ \\ 
					& CorgiPuppy/ \\
					& info-sys-admin-labs \\
					Принял: & Митричев Иван Игоревич \\
					Дата сдачи: & 16.04.2025 \\
				\end{tabular}

			\end{flushright}

			\vspace*{6cm}
			\Large \textbf{Москва \\ 2025}
		\end{center}
	\end{titlepage}
	
	\tableofcontents	
	\thispagestyle{empty}
	\newpage

	\pagenumbering{arabic}
	
	\section*{Описание и выполнение задачи}
	\addcontentsline{toc}{section}{Описание и выполнение задачи}	
	\large

	\subsection*{Задание 1}
	\addcontentsline{toc}{subsection}{Задание 1}
	\large
	\begin{center}
		\textbf{Вариант 15}
	\end{center}
	\par
	Написать скрипт \textit{sed}, который заменяет в файле все теги \textit{<div>} на \textit{<p>}, а теги \textit{</div>} удаляет. Протестировать скрипт на различных файлах, показав, что поставленная задача решена верно.
	\lstset{style=mystyle}
	\lstinputlisting[language=Bash]{src/task1/main.sh}
% \begin{center}
% 	\includegraphics[width=500pt]{task1.png}
% \end{center}

	\subsection*{Задание 2}
	\addcontentsline{toc}{subsection}{Задание 2}
	\large
	\begin{center}
		\textbf{Вариант 32}
	\end{center}

	\par
	\lstset{style=mystyle}
	\lstinputlisting[language=Bash]{src/task2/main.sh}
\end{document}
